\documentclass[10pt,a4paper]{article}
\usepackage[utf8]{inputenc}
\usepackage[english]{babel}
\usepackage{graphicx}
\usepackage{amsmath}

\begin{document}

\noindent
{\Large Informacion de los graficos}\\

Todos los graficos que estan en este directorio corresponden al potencial de confinamiento 

\begin{equation}
  V(\rho, z) = \left\{ \begin{matrix}
                         -V_2 & \rho<a_{\rho}\, |z|\leq\frac{a_z}{2} \\
                         V_1  & \frac{a_z}{2}< |z| \leq \frac{a_z + b_z}{2}\\
                         0    &
                       \end{matrix}\right.
\end{equation}

\noindent
y el hamiltoniano $3D$ es

\begin{equation}
H = -\frac{\hbar^2}{2 m_e}\left[\frac{1}{\rho}\frac{\partial}{\partial \rho}\left(\rho \frac{\partial}{\partial \rho} \right)
+ \frac{1}{\rho^2}\frac{\partial^2}{\partial \varphi^2}+\frac{\partial^2}{\partial z^2} \right] + V(\rho, z)
+ \frac{1}{2} m_e \omega^2 \rho^2 - i \hbar \omega \frac{\partial}{\partial \varphi}\, ,
\end{equation}

Estudiamos primero el caso de un solo electron, para eso resolvimos el problema $3D$ y despues el $1D$ efectivo con dos
funciones de onda radial distinta, el estado fundamental de Landau y el estado fundamental del pozo.

Para el caso de dos particulas hicimos lo mismo para la funcion de onda espacial simetrica.

Todos los graficos estan hechos con los mismos parametros que son:
\begin{eqnarray*}
m_e &=& 0.041\,  \\
a_{\rho} &=& 7\,nm\\
a_z &=& 7\,nm \\
b_z &=& 2.5\, nm \\
V_1 &=& 0.37\, eV \\
V_2 &=& 0.10884\, eV
\end{eqnarray*}

\noindent
la masa efectiva corresponde al material $Ga_xIn_{1-x}As$
\end{document}
