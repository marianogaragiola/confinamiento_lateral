\documentclass[a4paper,10pt]{article}
\usepackage{inputenc}
\usepackage{graphicx}
\usepackage{amsmath}

%opening
\title{Effective Coulomb Potentials}
\author{FMP}

\begin{document}
\renewcommand{\labelenumi}{\bf \alph{enumi})}
\renewcommand{\labelenumii}{\alph{enumi}$_\arabic{enumii}$)}
\newcommand{\ve}[1]{\mathbf{#1}}
\newcommand{\prob}[1]{ \stepcounter{problema}\noindent{\bf Problema
\arabic{problema}:}{\it#1}}
\newcommand{\versor}[1]{\,\hat{\!#1}}
\newcommand{\titulo}[1]{\noindent\begin{Large}{\bfseries\underline{#1}}\end{
Large} \\}
\newcommand{\subtitulo}[1]{\noindent\begin{large}{$\bullet$\scshape
#1}\end{large} \\}
\newcommand{\ket}[1]{\left| #1 \right\rangle}
\newcommand{\bra}[1]{\left\langle #1 \right|}
\newcommand{\braket}[3]{\left\langle #1 \right| #2 \left| #3 \right\rangle}
\newcommand{\overlap}[2]{\left\langle #1 \right| \left. #2 \right\rangle}
\maketitle


\section*{Quasi 1d problem ( lateral confinement )}

Calculations to obtain the effective potentials for HO confinement of the 
Coulomb interaction between electrons. We Follow the work
in\cite{bednarek_effective_2003}. 

Consider the following 2 electron Hamiltonian,


\begin{equation}
H(\ve{r}_1, \ve{r}_2) = h(\ve{r}_1) + h(\ve{r}_2) +
\frac{\eta}{\left|\ve{r}_1-\ve{r}_2\right|}
\end{equation}

where,

\begin{equation}
h(\ve{r}_i) = -\frac{\hbar^2}{2 m_i}\nabla^2_{x_i y_i} + V_{conf}(x_i,y_i) - \frac{\hbar^2}{2
m_i}\frac{\partial^2}{\partial z_i^2} + V_{long}(z_i)
\end{equation}

and,

\begin{eqnarray}
\eta &=& \frac{1}{\kappa} \\
V_{conf}(x_i,y_i) &=& \frac{1}{2}m\omega^2(x^{2}_i + y^{2}_i) \\
V_{long}(z_i) &=& - V_L e^{-b_L (z_i + R/2)^2} - V_R e^{-b_R (z_i - R/2)^{2}}.
\end{eqnarray}

We want to obtain a Mean field Hamiltonian using the HO wave functions to deduce
an effective interaction between electrons. The effective Hamiltonian is to be used 
for a scattering process in which both electrons always in the ground state of the
confinement. The WF of the system can then
be approximated as

\begin{equation}
\Psi(\ve{r}_1,\ve{r}_2) = \phi_0(x_1,y_1)\phi_0(x_2,y_2)\psi(z_1,z_2).
\end{equation}

with $\phi_0$ the ground state of the oscillator,

\begin{equation}
\phi_0(x,y)=c_{\omega}\sqrt{\frac{2}{\pi}}e^{-(x^2+y^2)c_{\omega}^2}
\end{equation}

\noindent where $c_{\omega}=\sqrt{\frac{m \omega}{2\hbar}}$.

The Hamiltonian expectation value for this state is

\begin{equation}
\label{emvh}
\left\langle H(\ve{r}_1, \ve{r}_2)\right\rangle_\Psi =
\left(2\hbar\omega -\sum_{i=1,2} \left\langle \frac{\hbar^2}{2
m_i}\frac{\partial^2}{\partial z_i^2}+ V_{long}(z_i)\right\rangle_{\psi(z_1,z_2)}\right)+\eta \left\langle\frac{1}{\left|\ve{r}_1-\ve{r}_2\right|}\right\rangle_\Psi.
\end{equation}

Since $|\phi_0|^2\equiv 1$, one can see that the only term that couple $z$ with $x$ and
$y$ is the Coulomb interaction. We want to obtain an expresion for this term in the form

\begin{eqnarray}\label{coulint}
W_{12}&=&\int d\ve{r}^3_1 \int d\ve{r}^3_2\frac{|\Psi(\ve{r}_1,\ve{r}_2)|^2}{r_{12}} \\
&=& \int dz_1\int dz_2 |\psi(z_1,z_2)|^2\ V_{eff}(|z_1-z_2|),
\end{eqnarray}

\noindent which would allow us to calculate the effective potential $V_{eff}(|z_1-z_2|)$
to be used in the Hamiltonian with reduced dimensionality. 

\section{Calculation}

First we define some quantities that we need to use. The Fourier transform of the
confinement wave function

\begin{eqnarray}
\rho_0(k_x,k_y)&=&\mathcal{F}\left\lbrace\phi^2_0\right\rbrace = \int
dx\int
dy\ \phi^2_0(x,y)e^{i(k_x x + k_y y)} \\
&=& e^{-(k_x^2+k_y^2)\ l^2/4}
\end{eqnarray}

\noindent and we can offcourse write $\phi^2_0$ as the inverse transform of $\rho_0$,

\begin{equation}
\phi_0^2(x,y) = \mathcal{F}^{-1}\left\lbrace\rho_0\right\rbrace =\frac{1}{(2\pi)^2} \int dk_x \int dk_y \rho_0(k_x,k_y) e^{-i(k_x x + k_y y )}.
\end{equation}

The relation between $c_w$ and $l$ is

\begin{equation}
l=\sqrt{\frac{\hbar}{m \omega}}=\frac{1}{\sqrt{2}c_\omega}
\end{equation}

The same transformation is valid for the $z$ variables,

\begin{eqnarray}
|\psi(z_1,z_2)|^2 &=& \mathcal{F}^{-1}\left\lbrace\rho(q_1,q_2)\right\rbrace \\
 &=& \frac{1}{(2\pi)^2} \int dq_1 \int dq_2\ \rho(q_1,q_2) e^{-i(q_1 z_1 + q_2 z_2 )} \\
\end{eqnarray}

Instead of using the Coulomb potential, we use a Yukawa potential of the form
$\frac{e^{-\alpha r_{12}}}{r_{12}}$ an then let $\alpha \rightarrow 0$.

We begin by writing the potential as a Fourier transform of its inverse Fourier transform,

\begin{equation}
\frac{e^{-\alpha r_{12}}}{r_{12}} = \frac{1}{2\pi^2} \int d^2k \int dq \frac{e^{-i\left[k_x(x_1-x_2) + k_y(y_1-y_2) + q (z_1-z_2)\right]}}{k^2 + q^2 + \alpha^2}
\end{equation}

\noindent where $k^{2}=k^{2}_{x} + k^{2}_{y}$. Using this in eq.~(\ref{coulint}) we get,

\begin{equation}
W^{\alpha}_{12}= \frac{1}{2\pi^2}\int d^3\ve{r}_1 \int d^3\ve{r}_2 \int d^2k \int dq \frac{
|\phi_0(1)|^2|\phi_0(2)|^2|\psi(z_1,z_2)|^2 e^{-i\left[k_x(x_1-x_2) +
k_y(y_1-y_2) + q (z_1-z_2)\right]}}{k^2 + q^2 + \alpha^2}
\end{equation}

We now use the Fourier expression to perform the integration over the real variables. For
$x$ and $y$ is just replacement by the Fourier transform and the
only difference comes from the $z_1$ and $z_2$ integration,

\begin{equation}
W^{\alpha}_{12}= \frac{1}{2\pi^2}\int d^2k \int dq
\frac{\rho_0(k_x,k_y)\rho_0(k_x,k_y)\rho(-q,q) }{k^2 + q^2 + \alpha^2}
\end{equation}

\begin{eqnarray}
&=& \frac{1}{2\pi^2}\int d^2k \int^\infty_{-\infty} dz_1 \int^\infty_{-\infty} dz_2 
\ \rho^2_0(k_x,k_y)|\psi(z_1,z_2)|^2 \int^\infty_{-\infty} dq
\frac{e^{iq(z_1-z_2)}}{k^2 + q^2 + \alpha^2} \nonumber\\
&=& \frac{1}{2\pi^2}\int d^2k \int^\infty_{-\infty} dz_1 \int^\infty_{-\infty} dz_2 
\ \rho^2_0(k_x,k_y)|\psi(z_1,z_2)|^2 I(k,\alpha , z_1-z_2).\label{coulint2}
\end{eqnarray}

To solve the integral $I$ we notice that the denominator has two poles in the imaginary
plane $q_{\pm}=\pm i \sqrt{k^2+\alpha^2}$. They are opposed with respect to the point
$(0,0)$, so we can do the integral by enclosing one of them in a contour integration.
Then the residues in the poles give

\begin{eqnarray}
I(k,\alpha,z_1-z_2) &=& \int^\infty_{-\infty} dq \frac{e^{iq(z_1-z_2)}}{(q+i\sqrt{k^2+\alpha^2})(q-i\sqrt{k^2+\alpha^2})} \\
&=& \pi \frac{e^{-\sqrt{k^2+\alpha^2}|z_1-z_2|}}{\sqrt{k^2+\alpha^2}}.
\end{eqnarray}

Using this result in eq.~\ref{coulint2} we get

\begin{equation}\label{coul_veff}
W^{\alpha}_{12}=\int^\infty_{-\infty} dz_1 \int^\infty_{-\infty} dz_2 |\psi(z_1,z_2)|^2
\underbrace{\int \frac{d^2k}{2\pi} \rho^2(k_x,k_y) \frac{e^{-\sqrt{k^2+\alpha^2}|z_1
-z_2|}}{\sqrt{k^2+\alpha^2}}}_{V_eff(z_1-z_2)}.
\end{equation}

The integral in $V_{eff}$ can be solved for the case $\rho=e^{-(k_x^2+k^2_y)l^2/4}$. The
effective potential, on the other hand, depends on the confinement density as expressed
in eq.~\ref{coul_veff}. This density can be changed, according to the confinement. The
result for the oscillator confinement is

\begin{equation}
\label{V_eff}
V_{eff}(|z_1-z_2|)= e^{y^2+x^2}\sqrt{\frac{\pi}{2}}\frac{1}{l}\left(1-erf(x+y)\right)
\end{equation}

\noindent with $x=\frac{|z_1-z_2|}{\sqrt{2}l}$ and $y=\frac{\alpha l}{\sqrt{2}}$. For the Coulomb potential $y\equiv 0$.

Sometimes it is desirable to have an atractive or repulsive fixed charge. This can be done by adding a term of the form $\frac{1}{|\ve{r}-\ve{R}_0|}$. The effective potential
for this term is

\begin{equation}
V_{eff}(|z+R/2|) = \sqrt{\pi}\frac{1}{l}e^{\hat{x}^2}(1-erf(\hat{x}))
\end{equation}

\noindent with $\hat{x}=\frac{|z+R/2|}{l}$.

\section{Asymptotics}

Take into account the relations,

\begin{eqnarray}
l&=&\sqrt{\frac{\hbar}{m \omega}}=\braket{\phi_{0}}{x^2}{\phi_{0}} \\
x&=&\frac{|z_1-z_2|}{\sqrt{2}l}
\end{eqnarray}

\noindent for the case $x\rightarrow \infty$, we get

\begin{equation}
V_{eff}(|z_1-z_2|)\rightarrow \frac{1}{|z_1-z_2|} - \frac{l^2}{|z_1-z_2|^3} + \cdots
\end{equation}

\noindent and for $x\approx 0$,

\begin{equation}
V_{eff}(|z_1-z_2|)\approx \frac{1}{l}\sqrt{\frac{\pi}{2}} - \frac{|z_1-z_2|}{l^2}+ \cdots
\end{equation}

\section{Three dimensional quantum dots}

Lets us assume that the $V_{long}(z_{i})$ are potentials that represent aquantum dot
inside a nanowire. Since the dot shape is three dimensional, we should write the
QD potential as an spherical QD,
\begin{eqnarray}
 V_{QD}(\vec{r}_{i}) &=& -V e^{-b r^{2}_{i}} \\
 &=& -V e^{-b z^{2}_{i}}\,e^{-b\rho_{i}^{2}} \\
 &=& V_{long}(z_{i})\,V_{\perp}(\rho_{i})
\end{eqnarray}

with this expression we can rewrite Eq.~(\ref{emvh}) as,
\begin{equation}
\label{emvh}
\left\langle H(\ve{r}_1, \ve{r}_2)\right\rangle_\Psi =
\left(2\hbar\omega -\sum_{i=1,2} \left\langle \frac{\hbar^2}{2
m_i}\frac{\partial^2}{\partial z_i^2} \right\rangle_{\psi(z_1,z_2)} + 
\left\langle V_{long}(z_i)\right\rangle_{\psi(z_1,z_2)}\left\langle V_{\perp}(\rho_i)\right\rangle_{\phi_{0}(x,y)}\right) +
\eta \left\langle\frac{1}{\left|\ve{r}_1-\ve{r}_2\right|}\right\rangle_\Psi.
\end{equation}

\noindent where,

\begin{eqnarray}
 \left\langle V_{\perp}(\rho_i)\right\rangle_{\phi_{0}(x,y)} &=& \frac{2}{\pi}c^{2}_{\omega}\int^{\infty}_{-\infty} e^{-(b+2c_{\omega}^{2})(x^2 + y^2)} \textrm{d}x \textrm{d}y \\
 &=& \frac{1}{l^{2} b + 1}.
\end{eqnarray}

In this way, the modification that enters into the equation es that the 
coupling of the QD depth is now field dependent $V\rightarrow\frac{V}{l^{2} b + 1}$.
Be aware that this expression is valid only if $l\ll \frac{1}{\sqrt{b}}$, because we assumed
that the confinement wave function is given by the oscillator ground state, a fact that is not true
if the confinement by the QD is stronger.


\section{Comment on units}

In Gaussian units, the cynclotron frequency is
\begin{equation}
\omega = \frac{qB}{mc}
\end{equation}
\noindent
With this, the parameter $l$ in the effective potential is
\begin{equation}
l = \sqrt{\frac{\hbar c}{q B}}
\end{equation}

We can define the constant $\alpha$ such that
\begin{equation}
B = \frac{\alpha}{l^2}
\end{equation}
\noindent
So $\alpha = \frac{\hbar c}{q}$

We want a relation between $B$ and $l$ such that $B$ is measured in Tesla ($T$) and
l in nanometers ($nm$). For this purpose the value of $\alpha$ is:
\begin{equation}
\alpha = 658.4092645439 \, \ T nm^2
\end{equation}

\begin{figure}[h]
\begin{center}
 \includegraphics[scale=0.5]{B_vs_l.eps}
\end{center}
\end{figure}

Aca agrego lo que vendria a ser los niveles de landau para un electron en un campo magnetico $B$.

Los niveles de landau son:
\begin{equation}
\varepsilon_n = \hbar \omega_c \left(n + \frac{1}{2}\right)
\end{equation}

Queremos una relacion de la forma
\begin{equation}
\varepsilon_n = \beta \left(n+\frac{1}{2}\right) B
\end{equation}
\noindent
donde $\beta = \frac{\hbar q}{m c}$ esta en $\frac{eV}{T}$ y $B$ en $T$. 

Usando la masa reducida de $GaAs$, $m^* = 0.063 m_0$ tenemos $\beta = 1.83758049186099\times 10^{-3}$

\begin{figure}[h]
\begin{center}
 \includegraphics[scale=0.5]{landau_levels.eps}
\end{center}
\end{figure}

\section{limites}
El hamiltoniano unidimensional es:
\begin{equation}
\label{H_unidimensional}
H = h(1)+h(2)+V_{eff}(z_1,z_2)
\end{equation}
donde $V_{eff}(z_1,z_2)$ esta dado por la ec (\ref{V_eff}) y:

\begin{equation}
\label{h_1particula}
h(i) = -\frac{1}{2m_i}\frac{d^2}{dz_i^2}-\frac{V_0}{l^2 b+1}e^{-\frac{z_i^2}{2\sigma^2}}
\end{equation}

Como dijimos antes, todo esto vale para el caso $l\ll \sqrt{2}\sigma$. Usando que 
$l = \sqrt{\frac{\alpha}{B}}$ tenemos que $\frac{\alpha}{B}\ll 2\sigma^2$.
Por lo tanto la aproximaci\'on vale si:
\begin{equation}
\label{condicion}
B\gg\frac{\alpha}{2\sigma^2}
\end{equation}

Por ejemplo, para el caso $\sigma = 5nm$, tenemos $B\gg 13.168 T$.



\bibliography{bibliography}
\bibliographystyle{unsrt}



\end{document}
