\documentclass[10pt,a4paper]{article}
\usepackage[utf8]{inputenc}
\usepackage[english]{babel}
\usepackage{amsmath}
\usepackage{amsfonts}
\usepackage{amssymb}
\usepackage{makeidx}
\usepackage{graphicx}
\usepackage{lmodern}
\begin{document}
Algunas anotaciones sobre el problema con el fede.

El hamiltoniano para cada electr\'on es:
\begin{equation}
h = -\frac{\hslash^2}{2m}\frac{d^2}{dz^2}+V(z)
\end{equation}

El hamiltoniano para los electrones es:
\begin{equation}
H(z_1, z_2)=h(z_1)+h(z_2)+W(z_1,z_2)
\end{equation}

donde:
\begin{equation}
V(z) = -V_0 e^{-b(z-z_0)^2}
\end{equation}
\begin{equation}
W(z_1,z_2) = e^{y^2+x^2} \sqrt{\frac{\pi}{2}}\frac{1}{l}(1-erf(x+y))
\end{equation}
$x = \frac{|z_1-z_2|}{\sqrt{2}l}$ y $y=\frac{\alpha l}{\sqrt{2}}$

Ahora vamos a hacer complex rotation. Por lo tanto tenesmos que $z\rightarrow ze^{i\theta}$, 
por lo tanto el hamiltoniano de una particula queda:
\begin{equation}
h = -\frac{\hslash^2}{2m} e^{i2\theta} \frac{d^2}{dz^2}+V(ze^{i\theta})
\end{equation}
\begin{equation}
V(ze^{i\theta}) =-V_0e^{-bz^2cos\theta}\left[cos(bz^2 sin(2\theta))-i sin( bz^2 sin(2\theta))\right]
\end{equation}



\end{document}
